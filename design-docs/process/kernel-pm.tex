\documentclass[a4paper]{article}
\usepackage{zed-cm}
%\usepackage{graphicx}
%\markboth{Draft}{Draft}
\pagestyle{myheadings}
\begin{document}
\parskip 6 pt
\parindent 0 pt

\title{Kernel Process Model Requirements (v0.1)}
\author{Glyn Normington}
\maketitle
% The following three commands ensure the title page is stamped as
% confidential without a page number. Page numbering is started at the
% table of contents.
\thispagestyle{myheadings}
\pagenumbering{roman}
\setcounter{page}{0}

%=============================================================================
The Hebrides release of the SpringSource dm Server will be based on a separate \textit{kernel} component.
This document explores the requirements for how concurrent processes are modelled in the kernel.
\clearpage
\pagenumbering{roman}
\tableofcontents

% Type checking hacks
\newcommand{\true}{true}
\newcommand{\false}{false}
\renewcommand{\empty}{\emptyset}

%=============================================================================
\clearpage
\pagenumbering{arabic}
\section{Overview}

There are three layers in the kernel process model: events, processes, and a completion service that
provides callbacks when processes change state.

%=============================================================================
\clearpage
\section{Events}

Events have string names and associated value objects
\begin{zed}
[EventName, Value]
\end{zed}

Each event has a name and an optional value object.
\begin{schema}{Event}
name : EventName \\
value : Value \\
\end{schema}

Events are correlated using correlator objects.
\begin{zed}
[Correlator]
\end{zed}
\begin{schema}{EventWait}
 awaited : \power (EventName \cross Correlator) \\
 fired : EventName \cross Correlator \pfun Event \\
\where
 awaited \cap \dom fired = \emptyset \\
  \forall n : EventName, c : Correlator | (n, c) \in \dom fired @\\
 \t1 fired(n, c).name = n \\
\end{schema}

Events can be awaited. If the event has not already fired, the event is added to the awaited set of
events (and the await operation blocks until the event fires).
\begin{schema}{AwaitNotFired}
  \Delta EventWait \\
  c? : Correlator \\
  n? : EventName \\
 \where
  (n?, c?) \notin \dom fired \\
  awaited' = awaited \cup \{ (n?, c?) \} \\
  fired' = fired \\
 \end{schema}
 
 If the event has already fired, the state is not changed and the operation returns immediately.
 \begin{schema}{AwaitFired}
  \Xi EventWait \\
  c? : Correlator \\
  n? : EventName \\
 \where
  (n?, c?) \in \dom fired \\
 \end{schema}
 
 Awaiting an event covers either case.
 \begin{zed}
  Await \defs AwaitNotFire \lor AwaitFired \\
 \end{zed}

Correlators can be used to clean up old events. By definition, old events are not being waited upon.
\begin{schema}{Cleanup}
  \Delta EventWait \\
  c? : Correlator \\
 \where
  \forall a : awaited @ c? \neq second~a \\
  fired' = (EventName \cross \{c? \}) \ndres fired \\
 \end{schema}

\subsection{Serviceability Requirements}

Event waits and event firings will appear in the trace.

The set of awaited events and the collection of fired events will appear in the dump.

%=============================================================================
\clearpage
\section{Processes}

Process states and transitions are defined by a process schema.

Process schemas may be checked statically for properties such as deadlock freedom.

Child processes can be related to their parent processes?

Progress of a process relative to its schema to be visible in dump and trace.

Overall process graph to be deducible from the dump.

%=============================================================================
\clearpage
\section{Completion Service}

TBD.

\end{document}
