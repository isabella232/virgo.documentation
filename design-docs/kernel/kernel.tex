\documentclass[a4paper,12pt]{article}
\usepackage{zed-cm}
\usepackage{graphicx}
%\markboth{Draft}{Draft}
\pagestyle{myheadings}
\begin{document}
\parskip 6 pt
\parindent 0 pt

\title{dm Kernel 2.0 (v0.8)}
\author{Glyn Normington}
\maketitle
% The following three commands ensure the title page is stamped as
% confidential without a page number. Page numbering is started at the
% table of contents.
\thispagestyle{myheadings}
\pagenumbering{roman}
\setcounter{page}{0}

%=============================================================================
The Jersey release of the SpringSource dm Server is based on a separate \textit{kernel} component.
This document provides a partial formal model of the kernel.
\clearpage
\pagenumbering{roman}
\tableofcontents

% Type checking hacks
\newcommand{\true}{true}
\newcommand{\false}{false}
\renewcommand{\empty}{\emptyset}

%=============================================================================
\clearpage
\pagenumbering{arabic}
\section{Overview}

The dm Kernel is an embeddable OSGi component that manages various kinds of artefacts which may be obtained via a repository.

The embedding features of the kernel are discussed in the next section after which installation and
uninstallation of artefacts, the runtime artefact model, and start and stop of artefacts are specified.

%=============================================================================
\clearpage
\section{Embedding}

Since the kernel may be \textit{embedded} in a larger program, it does not have sole
control of the OSGi framework on which it runs.

This is a helpful perspective even for programs like dm Server which 
use the kernel but which do not operate directly on the OSGi framework, since application code installed in dm Server
is free to operate directly on the OSGi framework outside the control of the kernel.

To provide predictable overall behaviour, the kernel must operate on the OSGi
framework in a well defined, and preferably intuitively appealing, way.

In the absence of other programs operating directly on the OSGi framework, the kernel
presents a relatively clean abstraction to its users.
In the presence of such programs, the abstraction breaks down and users need to understand
the kernel's interaction with the OSGi framework.

So we need to take into account the \textit{embedded} and \textit{non-embedded} use cases in the design of the kernel.
As observed above, the non-embedded use case merges into the embedded use case as soon
as application code starts operating directly on the OSGi framework.
Thankfully, this is explicit in the design of application code and the choice of third party libraries, most of which were developed without OSGi in mind.
So the complexity of the kernel's behaviour increases in a ``pay as you go'' manner.

%=============================================================================
\clearpage
\section{Artefacts and the Runtime Artefact Model}

One of the primary functions of the kernel is to enable various kinds of \textit{artefact} to be installed
and started.
Artefacts include OSGi bundles, PAR files, WAR files, configurations, and \textit{plans}.
A plan refers to zero or more artefacts which are to be installed when the plan is
installed.
\begin{zed}
AType ::= BUNDLE | PAR | WAR | CONFIG | PLAN \\
\end{zed}

Certain types or arfefact may be installed in an OSGi framework.
\begin{zed}
OSGITYPES == \{ BUNDLE, WAR \} \\
\end{zed}

Artefacts have names and versions, although whether or how these and artefact types relate to the 
content of an artefact need not concern us here.
\begin{zed}
[Artefact, AName, AVersion] \\
\end{zed}

In this specification, we deliberately ignore the distinction between artefacts, their descriptions, and
transformed artefacts.

We abbreviate the artefact type, name, version triple.
\begin{zed}
TNV == AType \cross AName \cross AVersion
\end{zed}

We define a relation for checking that a type, name, version triple has a supported type.
%%inrel conformsTo
\begin{axdef}
\_ conformsTo \_ : TNV \rel (\power AType)
\where
\forall t : AType; n : AName; v : AVersion; ts : \power AType @ \\
\t1  ((t, n, v)~conformsTo~ts )\iff (t \in ts) \\
\end{axdef}

The kernel maintains a portion of the dependency graph known as the \textit{runtime artefact model}
and uses the OSGi framework to determine the remaining dependencies.
The runtime artefact model records the \textit{root artefacts} (the artefacts which have been explicitly installed, \textit{roots} for short), plan resolution decisions (which
artefact is chosen to satisfy each version range in a plan), and the bundles which are cloned
into a plan.

The runtime artefact model does not record which plan is chosen to satisfy each
\texttt{import-library} manifest header or which bundle is chosen to satisfy
each \texttt{import-bundle} manifest header since these headers result in package
dependencies which are managed by the OSGi framework.
This split is necessary to keep the dependency graph accurate.
For example, if a bundle in the
graph is updated and an OSGi \texttt{refreshPackages} operation is performed, the graph may change
significantly and it would be extremely difficult, if not impossible, for the kernel to maintain an accurate copy of the OSGi framework's dependencies.

The runtime artefact model also records the lifecycle state of each artefact in the model.
\begin{zed}
AState ::= INSTALLED | RESOLVED | STARTING | ACTIVE | \\
\t3 STOPPING | UNINSTALLING | UNINSTALLED \\
\end{zed}

An artefact which is in the $RESOLVED$, $STARTING$, $ACTIVE$, or $STOPPING$ state is
resolved.
In other words $STARTING$, $ACTIVE$, and $STOPPING$ can be regarded as sub-states of
$RESOLVED$.
\begin{zed}
RESOLVED\_STATES == \{ RESOLVED, STARTING, ACTIVE, \\
\t7 STOPPING \}
\end{zed}

An artefact which is in the $STARTING$, $ACTIVE$, or $STOPPING$ state is
active.
In other words $STARTING$ and $STOPPING$ can be regarded in some circumstances as
equivalent to $ACTIVE$.
\begin{zed}
ACTIVE\_STATES == \{ STARTING, ACTIVE, STOPPING \}
\end{zed}

The runtime artefact model has a set of supported artefact types, a collection of artefacts, each uniquely identified by type,
name, and version, a set of root artefacts, a dependency relation, a unique state for each artefact,
and set of resolved and actcive artefacts.
%% inrel dependsOn
\begin{schema}{RAM}
types : \power AType \\
art : TNV \pinj Artefact \\
roots : \power TNV \\
dependsOn : TNV \rel TNV \\
st : TNV \pfun AState \\
resolved : \power TNV \\
active : \power TNV \\
\where
\forall  tnv : \dom art @ tnv~conformsTo~types \\
roots \subseteq \dom art \\
\dom st = \dom art \\
dependsOn \subseteq \dom art \cross \dom art \\
resolved = st\inv \limg RESOLVED\_STATES \rimg \\
active = st\inv \limg ACTIVE\_STATES \rimg \\
dependsOn \limg resolved \rimg \subseteq resolved \\
dependsOn \limg active \rimg \subseteq active \\
\end{schema}
Each artefact in the model has a supported type.
Each root artefact is known in the model.
Some artefacts in the model depend on other artefacts in the model.
Each artefact has a state.
Resolved artefacts are those with a $RESOLVED$, $STARTING$, $ACTIVE$, or $STOPPING$ state.
Active artefacts are those with a $STARTING$, $ACTIVE$, or $STOPPING$ state.
Active artefacts may only depend on other active artefacts.

Each kernel instance has a runtime artefact model under its sole control.
Kernel instances running on an OSGi framework share the content of the framework.
These relationships are depicted in Figure \ref{fig:kernel-rams}.
\begin{figure}[h!]
\includegraphics*[scale=1]{kernel-rams.pdf}
\caption{Multiple kernels and runtime artefact models}
\label{fig:kernel-rams}
\end{figure}

%=============================================================================
\clearpage
\section{Runtime Artefact Model Operations}

Initially a runtime artefact model has a set of supported artefact types and no contents.
\begin{schema}{RAMInit}
RAM' \\
types? : \power AType \\
\where
types' = types? \\
art' = \empty \\
\end{schema}

The set of root artefacts may be inquired.
\begin{schema}{RAMgetRoots}
\Xi RAM \\
roots! : \power TNV \\
\where
roots! = roots \\
\end{schema}

The set of supported artefact types may be inquired.
\begin{schema}{RAMgetArtefactTypes}
\Xi RAM \\
types! : \power AType \\
\where
types! = types \\
\end{schema}

An artefact may be inquired by type, name, and version.
\begin{schema}{RAMgetArtefact}
\Xi RAM \\
t? : AType \\
n? : AName \\
v? : AVersion \\
a! : Artefact \\
\where
(t?, n?, v?) \in \dom art \\
a! = art(t?, n?, v?) \\
\end{schema}

A set of artefacts may be inquired by type.
\begin{schema}{RAMgetArtefactsByType}
\Xi RAM \\
t? : AType \\
as! : \power Artefact \\
\where
as! = \{ tnv : \dom art | tnv~conformsTo~\{ t? \} @ art~tnv \} \\
\end{schema}

A set of artefacts may be inquired by type and name.
\begin{schema}{RAMgetArtefactsByTypeAndName}
\Xi RAM \\
t? : AType \\
n? : AName \\
as! : \power Artefact \\
\where
as! = \{ tnv : \dom art; v : AVersion | tnv = (t?, n?, v) @ art~tnv \} \\
\end{schema}

A set of artefacts may be inquired by type, name, and version range.
\begin{schema}{RAMgetArtefactsByTypeNameAndVersionRange}
\Xi RAM \\
t? : AType \\
n? : AName \\
vr? : \power AVersion \\
as! : \power Artefact \\
\where
as! = \{ tnv : \dom art; v : vr? | tnv = (t?, n?, v) @ art~tnv \} \\
\end{schema}

Operations that modify the runtime artefact model do not change the set of supported types.
\begin{schema}{RAMOp}
\Delta RAM \\
\where
types' = types \\
\end{schema}

A root may be installed into the model.
\begin{schema}{RAMInstallRootOk}
RAMOp \\
tnv? : TNV \\
a? : Artefact \\
\where
tnv? \notin \dom art \\
art' = art \cup \{ tnv? \mapsto a? \} \\
roots' = roots \cup \{ tnv? \} \\
dependsOn' = dependsOn \\
st' = st \cup \{ tnv? \mapsto INSTALLED \} \\
\end{schema}

Resolving a root involves a number of steps each of which operates on the model.

Artefacts may be installed to satisfy the transitive dependencies of roots.
\begin{schema}{RAMInstallOk}
RAMOp \\
tnv? : TNV \\
a? : Artefact \\
\where
tnv? \notin \dom art \\
art' = art \cup \{ tnv? \mapsto a? \} \\
roots' = roots \\
dependsOn' = dependsOn \\
st' = st \cup \{ tnv? \mapsto INSTALLED \} \\
\end{schema}

An individual dependency from one artefact in the model to another in the model may be recorded.
\begin{schema}{RAMRecordDependency}
RAMOp \\
from? : TNV \\
to? : TNV \\
\where
from? \in \dom art \\
to? \in \dom art \\
art' = art \\
roots' = roots \\
dependsOn' = dependsOn \cup \{ from? \mapsto to? \} \\
\end{schema}

A set of artefacts may have their state set to a given value.
This is an operation on a set of artefacts since, for example, marking an individual artefact's state as
resolved is insufficient for handling circular dependencies.
\begin{schema}{RAMSetStateOk}
RAMOp \\
res? : \power TNV \\
s? : AState \\
\where
res? \subseteq \dom art \\
art' = art \\
roots' = roots \\
dependsOn' = dependsOn \\
st' = st \oplus ( res? \cross \{ s? \} ) \\
\end{schema}

The following operation specifies the overall operation of installing and resolving a root
using an input collection of dependencies.
\begin{schema}{RAMResolveRootOk}
RAMOp \\
tnv? : TNV \\
a? : Artefact \\
deps? : TNV \pinj Artefact \\
\where
tnv? \notin \dom art \\
\also
art' = art \cup \{ tnv? \mapsto a? \} \cup deps? \\
roots' = roots \cup \{ tnv? \} \\
st ' = st \cup ((\{ tnv? \} \cup \dom deps?) \cross \{ RESOLVED \}) \\
\also
dependsOn \subseteq dependsOn' \\
\dom deps? \subseteq \ran dependsOn' \\
\end{schema}
The root must not already be in the model.
The root and new dependencies are added to the model in $RESOLVED$ state.
The new dependencies are depended upon in the updated model.

%=============================================================================
\clearpage
\section{Installing and Uninstalling Artefacts}

A \textit{repository} is a collection of artefacts each uniquely identified by type, name, and version.
\begin{zed}
Repository == TNV \pinj Artefact
\end{zed}

A kernel may be configured with a repository from which it can install artefacts.
A kernel which is not configured with a repository is equivalent, at least from an install perspective, to
a kernel configured with an empty repository.

So we model a kernel as having a runtime artefact model and a repository.
\begin{schema}{Kernel}
RAM \\
repo : Repository \\
\end{schema}

Kernel instances may share a repository or not, as shown in Figure \ref{fig:kernel-repos}.
\begin{figure}[h!]
\includegraphics*[scale=0.7]{kernel-repos.pdf}
\caption{Kernels and repositories}
\label{fig:kernel-repos}
\end{figure}

We model kernel kernel instance creation as taking a repository parameter.
\begin{schema}{KernelInit}
Kernel' \\
RAMInit \\
repo? : Repository \\
\where
repo' = repo? \\
\end{schema}

Thereafter, the repository associated with a kernel can change in arbitrary ways.
We summarise this by stating that the kernel has a notional operation which can update
the kernel's repository arbitrarily. This operation is not implemented.
\begin{schema}{UpdateRepository}
\Delta Kernel \\
\Xi RAM \\
repo? : Repository \\
\where
repo' = repo? \\
\end{schema}

Kernel operations do not modify the repository.
\begin{schema}{KernelOp}
\Delta Kernel \\
\where
repo' = repo \\
\end{schema}

Certain operations on the runtime artefact model are promoted into corresponding kernel operations
(although these will probably be grouped in a \texttt{RuntimeArtifactModel}\footnote{American spelling is used in Java interfaces and javadoc.} interface).
\begin{zed}
getRoots \defs RAMgetRoots \land KernelOp \\
\also
getArtefactTypes \defs RAMgetArtefactTypes \land KernelOp \\
\also
getArtefact \defs RAMgetArtefact \land KernelOp \\
\also
getArtefactsByType \defs RAMgetArtefactsByType \land KernelOp \\
\also
getArtefactsByTypeAndName \defs \\
\t1 RAMgetArtefactsByTypeAndName \land KernelOp \\
\also
getArtefactsByTypeNameAndVersionRange \defs \\
\t1 RAMgetArtefactsByTypeNameAndVersionRange \land KernelOp \\
\end{zed}

Since the contents of the repository may change during the lifetime of the kernel, we can think of a
repository as being an input parameter to the install operation.
Invariant relationships between the contents of the repository and the contents of the
OSGi framework can only be maintained by a system which controls all the kernel instances
sharing the OSGi framework and all the repositories\footnote{Retracting an artefact from a repository
may cause the artefact's URI, that may have been previously obtained by a kernel, to become invalid and produce runtime errors in the kernel if the URI is used.}
associated with those kernel instances.
Any such invariants will also depend on the behaviour of any bundles which operate
directly on the OSGi framework.

An install operation on the kernel takes a single root artefact.
This may be supplied directly or as a reference to an artefact in the kernel's repository.
The latter form is modelled below.
\begin{schema}{InstallOkRaw}
KernelOp \\
RAMResolveRootOk \\
\where
tnv? \in \dom repo \\
deps? \subseteq repo \\
a? = repo~tnv? \\
tnv? \in resolved' \\
\end{schema}
The supplied artefact is installed \textit{and} resolved.

\begin{zed}
InstallOk \defs InstallOkRaw \hide (a?, deps?)
\end{zed}

Artefacts may have dependencies on other artefacts.
For example, a plan may refer to a configuration file and a bundle may import a package which
is exported by another bundle.
Dependencies may be \textit{optional} or \textit{mandatory}.
For example, an import of a package which specifies the directive \texttt{resolution:=optional}
expresses an optional dependency on some bundle which exports that package.

When the kernel installs a root, it attempts to satisfy the dependencies of the root.
Some of these dependencies may be satisfied by artefacts that were previously installed.
The kernel attempts to satisfy any remaining dependencies by installing artefacts
from the kernel's repository.
These artefacts may also have dependencies which the kernel will attempt to satisfy, and so on.
A root's dependencies, together with their dependencies, and so on are referred to collectively
as the \textit{transitive dependencies} of the root. 
If any \textit{chain} of mandatory dependencies cannot be satisfied, then the installation of the
root fails.

If the kernel does not have a repository, only (directly supplied) roots may be installed in the kernel and, before a root may be installed successfully, the root's transitive dependencies must already be installed.

If the installation of a root fails, any artefacts which were installed to satisfy the root's transitive dependencies must remain installed in case they are now depended upon by some other artefact
that was installed concurrently and outside the control of the kernel.

To provide some level of clean-up after install failure, the transitive dependencies may be calculated and a check made that the OSGi bundles to be installed can resolve \textit{before} the root or any of its transitive dependencies are installed in the OSGi framework.
Since the OSGi framework may change concurrently with this calculation, the root installation
may still fail after (correct) validation beforehand.
This clean-up need not be fool-proof.

%=============================================================================
\clearpage
\section{Starting and Stopping Artefacts}

Installed artefacts may be started and later stopped any number of times and then, ultimately, uninstalled.
Certain installed artefacts may be refreshed to pick up changes.

The main complexity surrounds when to start and stop the transitive dependencies of a root.
Dependencies are those defined in the runtime artefact model and package dependencies
defined by the OSGi framework.
Service dependencies are too dynamic and unpredictable to be useful in this context.

\subsection*{Start}

In general, artefacts in a dependency graph should be started in \textit{reverse dependency order}.
This means that artefacts are started, whenever possible, after the artefacts on which they depend.
If there are cyclic dependencies, the artefacts in each cycle could be started in an arbitrary order,
although it may be better to choose a deterministic order such as the order of installation.

Dependencies of plans are started in an order defined by the plan.

When a root is installed, any implicit transitive dependency (that is, a dependency that need not
be defined in the runtime
artefact model) causes the artefact satisfying the dependency to be started \textit{before} the installation of the root completes.

Before a root is started, the kernel starts the transitive dependencies, in reverse dependency order.

Starting an artefact which is in \texttt{ACTIVE} state has no effect.

Starting a root starts all its dependencies.
\begin{schema}{RAMstartRootOk}
RAMOp \\
tnv? : TNV \\
\where
tnv? \in resolved \\
art' = art \\
roots' = roots \\
dependsOn' = dependsOn \\
st ' = st \oplus ((dependsOn\star \limg \{ tnv? \} \rimg) \cross \{ ACTIVE \}) \\
\end{schema}

\subsection*{Stop}

Before a root is stopped, the kernel stops certain transitive dependencies, in dependency order.

The kernel traverses the root's dependencies in the runtime artefact model and ignores dependencies in the OSGi framework. Also, the kernel 
does not stop transitive dependencies which are depended on, in the runtime artefact model,
by other active artefacts.

Stopping an artefact which is in \texttt{INSTALLED} or \texttt{RESOLVED} state has no effect.

Stopping a root stops all its unique dependencies.
\begin{schema}{RAMstopRootOk}
RAMOp \\
tnv? : TNV \\
\where
tnv? \in \dom art \\
st~tnv? = ACTIVE \\
art' = art \\
roots' = roots \\
dependsOn' = dependsOn \\
(dependsOn\star \limg \{ tnv? \} \rimg) \ndres st = (dependsOn\star \limg \{ tnv? \} \rimg) \ndres st' \\
st'~tnv? = RESOLVED \\
\end{schema}

\subsection*{Kernel Operations}

These operations on the runtime artefact model are promoted into corresponding kernel operations
(although these will probably be grouped in a \texttt{Artifact} interface).
\begin{zed}
startRootOk \defs RAMstartRootOk \land KernelOp \\
\also
stopRootOk \defs RAMstopRootOk \land KernelOp \\
\end{zed}


%=============================================================================
%\clearpage
%\section{Alternative Designs}
%
%\subsection{Starting and Stopping Artefacts}
%
%Two alternative approaches to starting and stopping artefacts were considered and rejected.
%
%\subsection{Incremental Start}
%The kernel keeps track of the transitive dependencies which were installed for each root and, when
%the root is started, starts only these transitive dependencies, in reverse dependency order.
%
%Unfortunately, if roots with overlapping dependency graphs are installed in succession, \textit{all}
%the roots installed prior to a given root may need to be started before the given root is started, which
%makes this approach unattractive.
%
%Stopping a root is also problematic: roots would essentially need to be stopped in reverse
%install order.
%
%\subsection{Immediate Start}
%The kernel keeps track of the transitive dependencies which were installed for each root and, before
%completing the installation of the root, starts the transitive dependencies in reverse dependency
%order.
%
%Unfortunately, root plans need to be treated as a special case.
%When a root plan is installed, only its indirect dependencies are started.
%Without this special case, the start operation on the root plan is useless.

\end{document}
