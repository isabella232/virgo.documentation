\documentclass[a4paper,9pt]{article}
\usepackage{zed-cm}
\usepackage{graphicx}
\markboth{Unclassified, Draft}{Unclassified, Draft}
\pagestyle{myheadings}
\begin{document}
\parskip 6 pt
\parindent 0 pt

\title{A Formal Model of OSGi R4 Modularity (v0.83)}
\author{Glyn Normington}
\maketitle
% The following three commands ensure the title page is stamped as
% confidential without a page number. Page numbering is started at the
% table of contents.
\thispagestyle{myheadings}
\pagenumbering{alph}
\setcounter{page}{0}

%=============================================================================
The aim is to model a subset of the features of the modularity layer of
OSGi Release 4.

%\clearpage
%\pagenumbering{roman}
%\tableofcontents

% Type checking hacks
\newcommand{\true}{true}
\newcommand{\false}{false}
%\renewcommand{\empty}{\emptyset}
\newcommand{\ModuleDefZero}{ModuleDef_0}
\newcommand{\ModuleDefOne}{ModuleDef_1}
\newcommand{\ModuleDefTwo}{ModuleDef_2}
\newcommand{\ModuleDefThree}{ModuleDef_3}
\newcommand{\ModuleDefFour}{ModuleDef_4}

%=============================================================================
\clearpage
\pagenumbering{arabic}
\section{Introduction}
\label{cha:intro}
\textit{Probably omit this from the paper, although there needs to be
`warning' of the formal stuff near the start of the paper and in the
abstract.}

This specification only models a subset of the modularity function of OSGi R4.
It does not model:
\begin{itemize}
\item `uses'
\item optional resolution
\item dynamic imports
\item export filters
\item require-module
\end{itemize}

%=============================================================================
\clearpage
\section{UML Overview}
\label{cha:uml}
\textit{Probably omit this from the paper.}

Figure \ref{fig:uml} shows the main artefacts of OSGi modularity, some of
which will be covered by the formal model.

\begin{figure}[h!]
\includegraphics*[scale=0.5]{uml.pdf}
\caption{UML Overview}
\label{fig:uml}
\end{figure}

There are three new concepts in the diagram which are purely to explain the
other ``classes'' which relate directly to RFC 79 manifest headers.

\begin{description}
\item[ExportSpec] is a piece of the ``export signature'' of a module which
is a generalisation of \texttt{Export-Package}.
What the diagram doesn't show is that \texttt{Require-Bundle} with
\texttt{visibility:=reexport}
also produces these abstract exports.

\item[ImportSpec] is a kind of generalisation of
\texttt{Import-Package} and \texttt{DynamicImport-Package}.
AbstractImport really does generalise \texttt{Import-Package}.
But \texttt{DynamicImport-Package} can, because of its wild-carded package
name, be thought of as a kind of macro or template which generates
(at class load time) ``dynamic import instances'', described next.

\item[DynamicImportInstance] is effectively an import which is derived, at
class load time, from a dynamic import. AbstractImport generalises
DynamicImportInstance.
\end{description}

The crux of the diagram is that there is a ``matches'' relationship between
AbstractImport and AbstractExport.
This relationship describes potential package wirings - RFC 79 has to cover
the cases when an import matches more than one export.


%=============================================================================
\clearpage
\section{Basic Types}
\label{cha:basics}
\textit{Insert this section in section 3 of the paper.}

Some basic types need defining.

\subsection*{Module Names}

Modules are identified within a module system by module name
and module version number.
\begin{zed}
  [MName, MVer]
\end{zed}

Module version, consisting of the name and version number of a module,
needs an abbreviation.
\begin{zed}
  MV == MName \cross MVer
\end{zed}

\subsection*{Packages}
We need the notion of packages (or, more precisely, package names)
and package version numbers.
\begin{zed}
  [Package, PVer]
\end{zed}

%The default package name is special.
%\begin{axdef}
%  DefaultPackage : Package
%\end{axdef}

Package version, consisting of the name and version number of a package,
needs an abbreviation.
\begin{zed}
  PV == Package \cross PVer
\end{zed}

We need the notion of (fully qualified) class names and classes.
\begin{zed}
  [ClassName, Class]
\end{zed}

Each class name belongs a unique package.
\begin{axdef}
  package : ClassName \fun Package
\end{axdef}

Certain classes belong to packages beginning with ``java.'' (and ``org.omg.'' etc.).
\begin{axdef}
  javaClasses : \power ClassName
\end{axdef}

\subsection*{Class Spaces and Consistency}

A class space is a collection of classes indexed by class name.
\begin{zed}
  ClassSpace == ClassName \pinj Class
\end{zed}

We say that two class spaces $s$ and $t$ are \textit{consistent} if
and only if

$s \cup t \in ClassSpace$

, i.e. $s$ and $t$ agree on the
class names they have in common.

\subsection*{Attributes}

Arbitrary attribute names and values will be used to provide a flexible
import/export matching mechanism.
\begin{zed}
[AName, AValue]
\end{zed}

We will use mappings of attribute names to values.
\begin{zed}
  AMap == AName \pfun AValue
\end{zed}

Module  names are sometimes used as attribute names.
\begin{axdef}
  MNameAttr : AName
\end{axdef}

%=============================================================================
\clearpage
\section{Simple Class Loaders}
\label{cha:simpleloaders}
\textit{Insert this section in section 3 of the paper.}

Before we continue to describe OSGi modularity, we model a simple class loader
which delegates to a parent class loader.
This will help to familiarise the reader with the notation and how class
loading is modelled.

A simple class loader has a local collection of class
definitions, typically implemented as a sequence of jar files on a file
system, but modelled here as a class space.
It also has a class space of classes which have been defined
(i.e.\ loaded locally) by the class loader and which are derived from
the loader's local collection of class definitions.
Finally it has a class space of all classes loaded by the class loader. 
We defer tying this to the the parent class loader.
\begin{schema}{SimpleLoader}
  definitions : ClassSpace \\
  defined : ClassSpace \\
  loaded : ClassSpace \\
\where
  defined = definitions \cap loaded \\
\end{schema}
The defined classes are those which have definitions and which are also loaded.
Note that this only makes sense because the child and parent definitions turn
out to be disjoint. If a class is loaded from identical class files by two
distinct class loaders, the resultant class objects are distinct. 

A simple loader initially has no loaded classes.
\begin{schema}{SimpleLoaderInit}
  SimpleLoader \\
\where
  loaded = \emptyset
\end{schema}

We now model a successful class load.
\begin{schema}{SuccessfulLoad}
  \Delta SimpleLoader \\
  cn? : ClassName \\
\where
  definitions' = definitions \\
  defined \subseteq defined' \\
  loaded \subseteq loaded' \\
  cn? \in \dom loaded' \\
\end{schema}
The definitions are preserved.
The collections of defined and loaded classes may not decrease.
The specified class is loaded.
Note that more than one class may be loaded, which
may be necessary if the specified class refers to other classes
which have not yet been loaded.

A \textit{bootstrap} class loader is at the top of the
parent-child hierarchy and therefore only loads classes it also defines.
We refer to such a class loader as a \textit{top loader}.
\begin{schema}{TopLoader}
  SimpleLoader \\
\where
  defined = loaded \\
\end{schema}

We shall see later that a simple loader together with its parentage
up to and including the bootstrap loader may be modelled as a
top loader.

With this in mind, we can express the relationship between a child
class loader and its parentage.

\begin{schema}{ParentalConstraint}
  child : SimpleLoader \\
  parentage : TopLoader \\
\where
  child.loaded \setminus child.defined \subseteq parentage.loaded \\
  (\dom child.defined) \cap (\dom parentage.definitions) = \emptyset \\
  child.definitions \cap parentage.definitions = \emptyset \\
\end{schema}
Classes which the child has loaded but not defined must have been
loaded by the parentage.
The child has not defined a class for which the parentage has a
definition.
The child and parentage definitions are disjoint which models the
fact that distinct class loaders produce distinct class
instances when they load from the same class bytecode.

The first two properties are ensured by the child class loader
first of all delegating each class load to its parent and then only
attempting to define the class if its parent failed to load the class.

It follows that $child.loaded$ and $parentage.loaded$ are consistent
class spaces.
The proof is omitted because of lack of space.

We now show how a child together with a top loader satisfying the
above constraint can be combined to produce a new top loader.

\begin{schema}{Combine}
  ParentalConstraint \\
  loader: SimpleLoader \\
\where
  loader.definitions = child.definitions \oplus parentage.definitions \\
  loader.defined = child.defined \cup parentage.defined \\
  loader.loaded = child.loaded \cup parentage.loaded \\
\end{schema}

The proof that the result is really a top loader is omitted for lack of space.
\begin{argue}
  Combine \vdash loader \in TopLoader
\end{argue}

The reverse process of factoring a child class loader from a top loader
is also sometimes possible, but is not explored further here.

%=============================================================================
\clearpage
\section{Delegating Class Loaders}
\label{cha:delegatingloaders}
\textit{Possibly omit this from the paper.}

As a further step towards OSGi R4 modularity, we extend the notion of
a simple loader to model imports and exports.

An importer is a simple loader which defines the class names it is willing
to import from elsewhere. In OSGi R4 as we shall see later, imports are
defined in terms of packages, but we will overlook that level of detail for
the moment. For simplicity, we also ignore the parent class loader.
\begin{schema}{Importer}
  SimpleLoader \\
  imports : \power ClassName \\
  imported : ClassSpace \\
\where
  imported = loaded \setminus defined \\
 \dom imported \subseteq imports \\
\end{schema}
Classes which are loaded, but not defined, are imported.
Imported classes must be named as imports.

An exporter is a simple loader which defines the class names it is willing
to export to an importer.
\begin{schema}{Exporter}
  SimpleLoader \\
  exports : \power ClassName \\
  exported : ClassSpace \\
\where
  exported \subseteq loaded \\
  \dom exported \subseteq exports \\
\end{schema}
Exported classes must be loaded and named as exports.

We now define a constraint between a `matching' importer and exporter.
\begin{schema}{DelegationConsistency}
  i : Importer \\
  e : Exporter \\
\where
  i.definitions \cap e.definitions = \emptyset \\
  \LET shared == i.imports \cap e.exports @ \\
\t1 shared \dres i.imported \subseteq shared \dres e.exported \\
\end{schema}
The importer and exporter definitions are disjoint which models the
fact that distinct class loaders produce distinct class
instances when they load from the same class bytecode.
If shared classes denote those which may be both imported by the importer and
exported by the exporter, then the shared classes which have been imported
have also been exported.

It follows that $shared \dres i.loaded$ and $shared \dres e.loaded$ are
consistent class spaces.

One of the main problems solved by OSGi R4 is how to match imports to
exports in order to maintain proper consistency of class spaces
and avoid certain kinds of type mismatches resulting in class loading
failures, class cast exceptions, etc.

The solution starts with a `module definition'.

%=============================================================================
\clearpage
\section{Module Definition}
\label{cha:moduledef}
\textit{Insert the start of section in section 4.2 (Module Content)
of the paper
and intersperse the import/export related schemas into sections
4.3 (Imported Packages) and 4.4 (Exported Packages).
Insert the `matches' relation in section 4.8 (Module Resolution) of the paper.}

A module system contains a collection of modules which share packages
in various ways.

Before we model the module system, we focus first on the definition of an
individual module and secondly, in the next section, on the runtime state of
an individual module.
A module's definition identifies the module by module  name and
module version.
It also contains class definitions which are contained
in the module and may be loaded by the module.

\begin{schema}{ModuleId}
  mname : MName \\
  mver : MVer \\
  mv : MV \\
\where
  mv = (mname, mver)
\end{schema}

\begin{schema}{\ModuleDefZero}
  ModuleId \\
  definitions : ClassSpace \\
\where
  mv = (mname, mver)
\end{schema}

A complete list of imported and exported classes would be hard to maintain.
Also a given class module does not typically want an arbitrary mixture of
classes from various sources.
The OSGi design point is to make the Java package the minimum unit of
resolution.

Each package which a module exports has an export specification which
specifies the module  name, module version, package version, and
a set of matching attributes some of which must be supplied by an importer
which wishes to use the exported package.
The export specification may also require a matching importer to specify
module  name or module version (or both).
\begin{schema}{ExportSpec}
  ModuleId \\
  pver : PVer \\
  attr : AMap \\
  mandatory : \power AName \\
\where
  MNameAttr \notin \dom attr \\
  mandatory \subseteq (\dom attr \cup \{ MNameAttr \}) \\
\end{schema}

Each package which a module imports has an import specification.

A module name may be specified (using a singleton set) or not (using an empty set).
Module versions may only be constrained if a module name has been specified.
A range of acceptable module versions or package versions is represented as a set containing
all the elements of the range.
\begin{schema}{ImportSpec}
  mname : \power MName \\
  mver : \power MVer \\
  pver : \power PVer \\
  attr : AName \pfun AValue \\
\where
  \# mname \leq 1 \\
  mname = \empty \implies mver = MVer \\
  MNameAttr \notin \dom attr \\
\end{schema}

An export specification matches an import specification if and only if:
\begin{itemize}
\item any module
 name, module versions, and package versions in the import specification are
satisfied by exported module  name, module version, and package version,
\item any arbitrary attributes specified in the import specification match the values specified
in the export specification,
\item any arbitrary attributes specified as mandatory in the export specification are specified
in the import specification,
\item if the export specification mandated the module  name, the import specification
specifies the module  name.
\end{itemize}
\begin{axdef}
  matches: ImportSpec \rel ExportSpec
\where
  \forall is : ImportSpec; es : ExportSpec @ \\
\t1 is \mapsto es \in matches \iff \\
\t2 is.mname = \empty \lor is.mname = \{ es.mname \} \land \\
\t2 es.mver \in is.mver \land \\
\t2 es.pver \in is.pver \land \\
\t2 is.attr \subseteq es.attr \land \\
\t2 es.mandatory \subseteq (\dom is.attr \cup \{ MNameAttr \}) \land \\
\t2 MNameAttr \in es.mandatory \implies  is.mname \neq \empty \\
\end{axdef}

A module's definition identifies the packages the module imports and exports
by providing corresponding import and export specifications.
A module may provide at most
one import specification for a given package name but it may have multiple
export specifications for a given package name.

A module may not name itself in an import specification.

However a module can import and export the same package, although
as we shall see later, such a package is defined by one and only one
module.
If a module imports and exports the same package,
either the import or the
export is honoured when the module is resolved and the other statement
is disregarded.
We add abbreviations for the sets of classes which may be exported
and imported.
\begin{schema}{ModuleDef}
  \ModuleDefZero \\
  exports : Package \rel ExportSpec \\
  imports : Package \pfun ImportSpec \\
  classExports, classImports : \power ClassName \\
\where
  \forall es : \ran exports @ es.mv = mv \\
  \forall is : \ran imports @ mv \notin is.mname \cross is.mver \\
  classExports = package \inv \limg \dom exports \rimg \\
  classImports = package \inv \limg \dom imports \rimg \\
\end{schema}

%% OSGi uses package names, wild-carded expressions, and
%% sequencing to build up rich sets of classes for sharing.
%% The model here simply describes the outcome.
%% $\dom imports$ and $\dom exports$ are the sets of \textit{potentially}
%% imported and exported classes, respectively.
%% We say \textit{potentially} because only classes which are actually
%% available for import or export will actually be imported or exported.

%% OSGi allows a class to be imported from any one of a number of modules.
%% So we model $imports$ as a relation.

%% OSGi allows a class to be exported globally or to one or more particular
%% (\textit{friend}) modules.
%% We allow for this by modelling $exports$ as a relation
%% between class names and module  names: a particular class name in
%% the domain of $exports$ would, in the case of a global export,
%% relate to every possible module  name whereas, in the case of
%% a more restrictive export, a class name would relate to fewer
%% module  names.

%=============================================================================
\clearpage
\section{Module}
\label{cha:module}
\textit{Insert this section in section 4.8 (Module Resolution) of the paper.}

Based on its definition, a module loads classes either from its contents
or by importing from another module\footnote{For the purposes of this
model, we ignore the parent class loader}. 
These loaded classes are modelled using various class spaces
in addition to the module definition:
\begin{description}
\item[defined] classes loaded by the module\footnote{Strictly speaking,
$defined$ models classes for which the module's class loader is the
\textit{defining} loader.},
\item[imported] classes imported from other modules.
\end{description}
Note that $defined$ and $imported$, together with the module's definition,
determine the other class spaces:
\begin{description}
\item[loaded] all classes available to a module,
\item[exported] classes exported to other modules,
\item[private] classes loaded by the module which are not exported.
\end{description}
\begin{schema}{Module}
  ModuleDef \\
  defined : ClassSpace \\
  imported : ClassSpace \\
  loaded : ClassSpace \\
  exported : ClassSpace \\
  private : ClassSpace \\
\where
  defined \subseteq definitions \\
  defined = classImports \ndres loaded \\
  imported = classImports \dres loaded \\
  exported = classExports \dres loaded \\
  private = classExports \ndres loaded \\
\end{schema}

Some interesting properties follow.
\begin{argue}
  Module \vdash \\
\t2 \langle defined, imported \rangle \partition loaded \land \\
\t2 \langle private, exported \rangle \partition loaded \land \\
\t2 package \limg \dom defined \rimg \cap \dom imports = \emptyset \land \\
\t2 defined = definitions \cap (loaded \setminus imported) \\
\end{argue}
The defined and imported class spaces partition the loaded class space.
The private and exported class spaces also partition the loaded class space.
No classes are loaded locally which belong to imported packages.
Locally loaded classes are precisely those which have local definitions
and which have been loaded but not imported.

The proofs of these properties is left as an exercise for the reader.

%=============================================================================
\clearpage
\section{Module System}
\label{cha:splat}
\textit{Insert this section in section 4.8 (Module Resolution) of the paper.}

To give an indication of how the pieces of the specification fit together,
we model a service
platform of installed modules ($m$)
some of which are resolved and with a wiring function ($wire$) wiring together
resolved modules for packages.
\begin{schema}{ModuleSystemBase}
  m : MV \pinj Module \\
  resolved : \power MV \\
  wire : MV \cross Package \pfun MV \\
\where
 \ran wire \subseteq resolved \\
 resolved \subseteq \dom m \\
 first~\limg \dom wire \rimg \subseteq resolved \\
 (\forall mv  : resolved | (m~mv).imports \neq \emptyset @ mv \in first~\limg \dom wire \rimg )
\end{schema}

Note that the package wiring for a given package $p$ is described by the
function $\lambda mv : MV @ wire~(mv, p)$.

Two constraints must be satisfied. The first is that imports must be
matched by exports.
\begin{schema}{ImportExportMatching}
  ModuleSystemBase \\
\where
  (\forall m1 : MV; p : Package | (m1, p) \in \dom wire @ \\
\t1 p \in \dom (m~m1).imports \land \\
\t1 p \in \dom ((m \circ wire)(m1, p)).exports \land \\
\t1 (\LET is == (m~m1).imports~p @ \\
\t2 (\exists es : (((m \circ wire)(m1, p)).exports) \limg \{ p \} \rimg @ \\
\t3 (is, es) \in matches ))) \\
\end{schema}

The second is that matched importers and exporters are consistent.
\begin{schema}{WiringConsistency}
  ModuleSystemBase \\
\where
  \forall i, e : MV; p : Package | (i, p) \mapsto e \in wire @ \\
\t1 \LET mi == m~i; me == m~e; c == package \inv \limg \{ p \} \rimg @ \\
\t2 c \dres mi.imported \subseteq c \dres me.exported \\ 
\end{schema}

Then a module system must satisfy both constraints.
\begin{zed}
  ModuleSystem \defs ImportExportMatching \land WiringConsistency
\end{zed}

\begin{schema}{ResolveOk}
 \Delta ModuleSystem \\
 m? : MV \\
 mdef? : ModuleDef \\
\where
 m? \notin \dom m \\
 m? \in resolved' \\
\end{schema}

%=============================================================================
\clearpage
\section{Class Search Order}
\label{cha:classsearchorder}
\textit{Insert this section in section 4.9 (Run-time Class Search Order)
of the paper.}

Although we have avoided modelling the parent class loader and Require-Bundle
so far, it is essential to see how these features combine into the final
class search order.
We add the parent class loader in, modelled as a top loader, and the
contribution of all required modules modelled as a class space.
We also model the search order explicitly as a class space.
\begin{schema}{CompleteModule}
  parent : TopLoader \\
  required : ClassSpace \\
  search : ClassSpace \\
  ModuleDef \\
  defined : ClassSpace \\
  imported : ClassSpace \\
  loaded : ClassSpace \\
  exported : ClassSpace \\
  private : ClassSpace \\
\where
  (classImports \cup classExports) \cap javaClasses = \emptyset \\
  classImports \cap \dom required = \emptyset \\
  loaded = (defined \cup required \cup imported) \\
\t2 \oplus (javaClasses \dres parent.loaded) \\
  search = ((classImports \ndres (definitions \oplus required)) \cup imported) \\
\t2 \oplus (javaClasses \dres parent.definitions) \\
  defined \subseteq definitions \\
  defined = ((classImports \cup javaClasses) \ndres loaded) \setminus required \\
  imported = classImports \dres loaded \\
  exported = classExports \dres loaded \\
  private = classExports \ndres loaded \\
\end{schema}
Neither imports not exports may specify packages beginning with ``java.''.
Required classes do not include any classes which are specified as imports.

Loaded classes consist of those defined by the module, those required from other modules,
those imported from other modules, but all overridden with the parent class loader's packages
which begin with ``java.''.

The search order reflects the parent class loader being searched for packages beginning with
``java.'', and then imported packages, and then required and defined classes which do not
belong to impored packages.

The defined classes are those which are loaded but not inherited from the parent, imported,
or required.
Imported classes are those which are loaded and which belong to imported packages.
Exported classes are those which are loaded and which belong to exported packages.
Private classes are those which are loaded but which do not belong to exported packages.
%=============================================================================
\newpage
\section{Z Notation}
\label{cha:znot}

\makeatletter % the following code is taken from Mike Spivey's zed.tex

\def\symtab{\setbox0=\vbox\bgroup \def\\{\cr}
        \halign\bgroup\strut$##$\hfil&\quad##\hfil\cr}
\def\endsymtab{\crcr\egroup\egroup
        \dimen0=\ht0 \divide\dimen0 by2 \advance\dimen0 by\ht\strutbox
        \splittopskip=\ht\strutbox \vbadness=10000
        \predisplaypenalty=0
        $$\halign{##\cr\hbox to\linewidth{% 
                \valign{##\vfil\cr
                        \setbox1=\vsplit0 to\dimen0 \unvbox1\cr 
                        \noalign{\hfil}\unvbox0\cr
                        \noalign{\hfil}}}\cr 
                \noalign{\prevdepth=\dp\strutbox}}$$
        \global\@ignoretrue}

\makeatother 

Numbers:
\begin{symtab}
        \nat & \verb/Natural numbers/ \{\verb/0,1,.../\} \\
%       \num & \verb/Integers (...,-1,0,1,...)/ \\
%       \nat_1 & \verb/Positive natural numbers/ \\
%       \upto & \verb/integral range/ \\
%       + & \verb/Addition/\quad\hfill 3 \\
%       - & \verb/Subtraction/\quad\hfill 3 \\
%       * & \verb/Multiply/\quad\hfill 4 \\
%       \div & \verb/Remainder/\quad\hfill 4 \\
%       \mod & \verb/Modulus/\quad\hfill 4 \\
%       < & \verb/Less than/ \\
%       > & \verb/Greater than/ \\
%       \leq & \verb/Less than or equal/ \\
%       \geq & \verb/Greater than or equal/ \\
%       \neq & \verb/Inequality/ \\
\end{symtab}
Propositional logic and the schema calculus:
\begin{symtab}
%       \lnot & \verb/Not/ \\
        \ldots\land\ldots & \verb/And/ \\
        \ldots\lor\ldots & \verb/Or/ \\
        \ldots\implies\ldots & \verb/Implies/ \\
%       \iff & \verb/If and only if/ \\
        \forall..\mid..\spot.. & \verb/For all/ \\
        \exists..\mid..\spot.. & \verb/There exists/ \\
%       \exists_1..\mid..\spot.. & \verb/There exists unique/ \\
        \ldots\hide\ldots & \verb/Hiding/ \\
%       \project & \verb/\project/ \\
%       \pre & \verb/\pre/ \\
%       \semi & \verb/\semi/
        \ldots\defs\ldots & \verb/Schema definition/ \\
        \ldots==\ldots & \verb/Abbreviation/ \\
        \ldots::=\ldots\mid\ldots & \verb/Free type definition/ \\
        \ldata\ldots\rdata & \verb/Free type injection/ \\
        [\ldots] & \verb/Given sets/ \\
        ',?,!,_0\ldots_9 & \verb/Schema decorations/ \\
        \ldots\shows\ldots & \verb/theorem/ \\
        \theta\ldots & \verb/Binding formation/ \\
        \lambda\ldots & \verb/Function definition/ \\
        \mu\ldots & \verb/Mu-expression/ \\
        \Delta\ldots & \verb/State change/ \\
        \Xi\ldots & \verb/Invariant state change/ \\
\end{symtab}
Sets and sequences:
%and bags:
\begin{symtab}
        \{\ldots\} & \verb/Set/ \\
        \{..\mid..\spot..\} & \verb/Set comprehension/ \\
        \power\ldots & \verb/Set of subsets of/ \\
%       \power_1 & \verb/Non-empty subsets of/ \\
%       \finset & \verb/Finite sets/ \\
%       \finset_1 & \verb/Non-empty finite sets/ \\
        \emptyset & \verb/Empty set/ \\
        \ldots\cross\ldots & \verb/Cartesian product/ \\
        \ldots\in\ldots & \verb/Set membership/ \\
        \ldots\notin\ldots & \verb/Set non-membership/ \\
        \ldots\cup\ldots & \verb/Union/ \\
        \ldots\cap\ldots & \verb/Intersection/ \\
        \ldots\setminus\ldots & \verb/Set difference/ \\
        \bigcup\ldots & \verb/Distributed union/ \\
%       \bigcap & \verb/Distributed intersection/ \\
        \#\ldots & \verb/Cardinality/ \\
%       \dcat & \verb/Distributed sequence concatenation/
        \ldots\subseteq\ldots & \verb/Subset/ \\
        \ldots\subset\ldots & \verb/Proper subset/ \\
        \ldots\partition\ldots & \verb/Set partition/ \\
        \seq & \verb/Sequences/ \\
%       \seq_1 & \verb/Non-empty sequences/ \\
%       \iseq & \verb/Injective sequences/ \\
        \langle\ldots\rangle & \verb/Sequence/ \\
%       \cat & \verb/Sequence concatenation/ \\
        \disjoint\ldots & \verb/Disjoint sequence of sets/ \\
%       \bag & \verb/Bags/ \\
%       \lbag\ldots\rbag & \verb/Bag/ \\
%       \inbag & \verb/Bag membership/ \\
\end{symtab}
%Here are the infix function symbols. Each symbol is
%shown with its priority:
%\begin{symtab}
%       \uplus & \verb/\uplus/ \\
%       \filter & \verb/Schema projection/ \\
%       \uminus & \verb/\uminus/
%\end{symtab}
Functions and relations:
\begin{symtab}
        \ldots\rel\ldots  & \verb/Relation/ \\
        \ldots\pfun\ldots & \verb/Partial function/ \\
        \ldots\fun\ldots  & \verb/Total function/ \\
        \ldots\pinj\ldots & \verb/Partial injection/ \\
        \ldots\inj\ldots  & \verb/Injection/ \\
%       \psurj & \verb/Partial surjection/ \\
%       \surj & \verb/Surjection/ \\
%       \bij  & \verb/Bijection/ \\
%       \ffun & \verb/Finite partial function/ \\
%       \finj & \verb/Finite partial injection/ \\
        \dom\ldots & \verb/Domain/ \\
        \ran\ldots & \verb/Range/ \\
        \ldots\mapsto\ldots & \verb/maplet/ \\
        \ldots\inv & \verb/Relational inverse/ \\
%       \ldots\plus & \verb/Transitive closure/ \\
        \ldots\star & \verb/Reflexive-transitive closure/ \\
%       \ldots\bsup n \esup & \verb/Relational iteration/ \\
        \ldots\limg\ldots\rimg & \verb/Relational image/ \\
%       \comp & \verb/Forward relational composition/ \\
%       \circ & \verb/Relational composition/ \\
        \ldots\oplus\ldots & \verb/Functional overriding/ \\
        \ldots\dres\ldots & \verb/Domain restriction/ \\
%       \ldots\rres\ldots & \verb/Range restriction/ \\
        \ldots\ndres\ldots & \verb/Domain subtraction/ \\
%       \ldots\nrres\ldots & \verb/Range subtraction/ \\
%       \id & \verb/Identity relation/ \\
\end{symtab}
Axiomatic descriptions:
%%unchecked
\begin{axdef}
  Declarations
\where
  Predicates
\end{axdef}
Schema definitions:
%%unchecked
\begin{schema}{SchemaName}
  Declaration
\where
  Predicates
\end{schema}
\end{document}
