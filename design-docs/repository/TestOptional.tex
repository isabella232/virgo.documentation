\documentclass[a4paper,titlepage,twoside,12pt]{article}
\usepackage{zed-cm}
\usepackage{graphicx}
\markboth{Draft}{Version 0.1}
\pagestyle{myheadings}
\begin{document}
\parskip 2 pt
\parindent 10 pt

\title{Testing Optional matters}
\author{Steve Powell}
\maketitle
% The following three commands ensure the title page is stamped as
% confidential without a page number. Page numbering is started at the
% table of contents.
\thispagestyle{myheadings}
\pagenumbering{roman}
\setcounter{page}{0}

%=============================================================================
We test the optional generic type (analogous to say, $\seq$) defined in \texttt{Optional.tex}.


%\clearpage
%\pagenumbering{roman}
%\tableofcontents

% Type checking hacks
\newcommand{\true}{true}
\newcommand{\false}{false}
\renewcommand{\emptyset}{\varnothing}
\newcommand{\defaultsTo}{\mathbin{\sf defaultsTo}}

%=============================================================================
\clearpage
\tableofcontents

\clearpage
\pagenumbering{arabic}
\section{Test}
We use $Opt$, $Nil$ and $Just$ to check that it type-checks with $f$UZZ.

Here is a given set:

\begin{zed}
	[BundleSymbolicName]
\end{zed}
and here is a state schema that uses it:
\begin{schema}{BundleState}
	bsn : Opt ~ BundleSymbolicName
\where
	(Nil \neq bsn) \implies (Take ~ bsn \in BundleSymbolicName) 
\end{schema}
The constraint merely states the minimum that can be true about the optional value. Since it type-checks with $f$UZZ then we can assume we have got the definitions approximately right.

\end{document}
